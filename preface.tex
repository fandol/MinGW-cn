% !Mode:: "TeX:UTF-8"
% !TEX TS-program = xelatex
% !TEX encoding = UTF-8 Unicode
% -*- coding: utf-8 -*-
% vim: set nobk noudf fdl=99 tw=75 fo+=Mm cc+=+2 :
%
% mingwstory: A simple handbook for MinGW/MSYS for Chinese users.
% preface.tex: prologue and preface for mingwstory.
%
% Copyright (C) 2014 LiTuX, the Fandol Team, all wrongs reserved.
%
%   Permission is granted to copy, distribute and/or modify this document
%   under the terms of the GNU Free Documentation License, Version 1.3
%   or any later version published by the Free Software Foundation;
%   A copy of the license is included in the package with the name LICENSE
%

\chapter{自序}
\label{ch:prologue}
我能够吞下玻璃而不伤害身体。

\clearpage{\thispagestyle{empty}\cleardoublepage}

\chapter{前言}
\label{ch:preface}

\begin{flushright}
    闹太套——小明
\end{flushright}

我终于把这个坑挖起来了。

若干年之前,我一定想不到将来的某一天,我竟然会不知天高地厚图样图森破的决定
挖这个坑,那时的我,头发还没有现在这么少……

接触到 MinGW 应该是 2008 年了。之前在 Linux 平台做过一点简单的开发,
而 Windows 平台还不得不使用陈旧的 VC 6.0,这对于一个已经逐渐习惯 vim,
喜欢从终端运行 make、喜欢 vim 的 quickfix 窗口的小白来说,
IDE 实在不够酷,不够屌。于是便开始了各种折腾,——比如从 VC 98、VC 2003、
VC 2005 提取出最小 C/C++ 开发环境(其实就是个精简版的 Windows SDK),
然后想尽办法让 cl 跟 vim 和平共处共同工作,甚至还通过 Wine 在 Linux 平台
用 nmake 和 cl 来“交叉”开发 Windows 应用……

之后,在 CUMCM 前夕,Chaosconst 同学想找 Windows 下的 gcc 用,而又不喜欢
Cygwin 版 gcc,于是找到了 MinGW 这个高端大气上档次的东东。
随着之后对它的了解和使用,越发觉得这是个足够狂拽炫酷叼炸天的东西,
所以尽管之后在 Windows 平台上的时间远超过 Linux 平台,
却一直坚守在 MinGW 平台,——一直过了五年。

这期间,MinGW 相比过去完善了许多,并且后来又发现了 MinGW--W64 这么给力的东
西。尽管使用过程中也遇到过各种不太对的地方,用一些 dirty trick 对付之,
或者提交 issue report……也算见证了 MinGW 的成长。

当然,必须要感谢许多其他先行者的博文日志。

于是我一直觉得,我也应该把自己折腾这些东西的过程中遇到的问题、如何对付等,
都贡献出来。而一直以来我都没有自己的电脑,也没有做过正经的个人网站,于是这
些“一直觉得”该做的事情,却只能一直放着,连个坑也只是在想法里面挖过,
没实际行动。

所以当 libcstl 群里的老隐童鞋提到弄这么一个教程的时候,
我觉得是时候开挖了,——我将不是一个人在战斗,而且这个“教程”会是有意义的。

其实说“教程”也并不合适,这个东西至少我在开挖的时候,定位更像是一组日志的整
理版,——尽管这个日志并不存在,或者说大部分存在于我的记忆中。
所以,我只能很没信心的说,我不敢保证这个东西的质量有多高,
毕竟这么多年以来,我还是个业余的开发者。
如果您有更好的建议或者发现什么问题,欢迎提出,也欢迎加入到文档的撰写中。

Welcome on board. :)

\begin{flushright}
    LiTuX,北京海淀

    2014 年 3 月 7 日\footnote{传说中的女生节}
\end{flushright}
\clearpage{\thispagestyle{empty}\cleardoublepage}

